\documentclass[a4paper,10pt]{article}
\usepackage{fontspec}
%\newfontfamily\mono{DejaVu Sans Mono}
%\makeatletter
%\def\verbatim@font{\mono\small}
%\makeatother
\usepackage[table]{xcolor}
%%%%%%%%%%%% Packages nécessaires à CmathLuaTeX %%%%%%%%%%%%%%%%%%%%%%%%%%%%%%
\usepackage{amsmath,amssymb,mathrsfs,mathtools}
\usepackage[e]{esvect} % pour de jolis vecteurs
\usepackage{stmaryrd} % pour les intervalles d'entiers
\usepackage{cancel} % pour biffer
\usepackage{tikz} % figures géométriques (nécessaire pour les tableaux de variations et de signes)
\usepackage{tkz-tab} % pour les tableaux de variations et de signes
\usepackage{array,cellspace} % pour les tableaux de valeurs
\newcommand{\bmmax}{2} % Évite de dépasser la limite TeX des 16 polices simultanées. Doit être chargée avant bm.
\usepackage{bm} % pour les formules en gras
%%%%%%%%%%%% Packages facultatifs %%%%%%%%%%%%%%%%%%%%%%%%%%%%%%
\usepackage{metalogo} % pour écrire un joli \LuaTeX ou \LuaLaTeX
\usepackage[scr=boondoxo,scrscaled=1]{mathalfa} % police mathscr moins penchée. \newcommand{\bmmax}{2} peut être effacé si ce package n'est pas utilisé.
%%%%%%%%%%%% Adobe Zapf Chancery font
\DeclareFontFamily{OT1}{pzc}{}
\DeclareFontShape{OT1}{pzc}{m}{it}{<-> s * [1.2] pzcmi7t}{}
\DeclareMathAlphabet{\mathpzc}{OT1}{pzc}{m}{it} % pour la police Zapf Chancery
%%%%%%%%%%%% Opérateurs
\DeclareMathOperator{\ch}{ch}
\DeclareMathOperator{\sh}{sh}
\renewcommand{\th}{\operatorname{th}}
\DeclareMathOperator{\argch}{argch}
\DeclareMathOperator{\argsh}{argsh}
\DeclareMathOperator{\argth}{argth}
\DeclareMathOperator{\Ima}{Im}
\DeclareMathOperator{\Ker}{Ker}
\renewcommand{\Im}{\operatorname{\mathfrak{Im}}}
\renewcommand{\Re}{\operatorname{\mathfrak{Re}}}
\DeclareMathOperator{\vect}{Vect}
\DeclareMathOperator{\pgcd}{pgcd}
\DeclareMathOperator{\ppcm}{ppcm}
%%%%%%%%%%%% CmathLuaTeX
\directlua{dofile('CmathLuaTeX.lua')}
\newcommand\Cmath[1]{\directlua{tex.print(Cmath2LaTeX('\detokenize{#1}'))}}
%%%%%%%%%%%%%%%%%%%%%%%%%%%%%%%%%%%%%%%%%%%%%%%%%%%%%%%%%%%%%%%%
\usepackage[margin=3cm]{geometry}
\usepackage{hyperref}
\usepackage{multirow}

\setlength{\parindent}{0cm}
\setlength{\cellspacetoplimit}{4pt}
\setlength{\cellspacebottomlimit}{4pt}

\definecolor{gris}{gray}{0.9}

\begin{document}

\begin{center}

{\Large Memento Cmath\LuaTeX}

{\large
\url{https://github.com/cdevalland/cmathluatex}

\href{mailto:christophe.devalland@ac-rouen.fr}{christophe.devalland@ac-rouen.fr}

29 décembre 2014}
\end{center}

-- \textbf{Raccourcis TeXworks}

Le confort d'utilisation de CmathLuaTeX sera optimale avec l'excellent éditeur TeXworks. Son apparence est rudimentaire mais il est d'une efficacité redoutable une fois qu'on a pris le temps de le personnaliser.

Les combinaisons de touches disponibles dans TeXworks sont :

\begin{tabular}{c p{12cm}}
Touches 	& Action \\
\hline
\verb?F9? 	&Compose la formule tapée juste avant le curseur avec la fonction \verb?$\Cmath{}$?. Action réversible en retapant F9.\\
\verb?Maj+F9? 	&Compose la formule tapée juste avant le curseur avec la fonction \verb?\[\Cmath{}\]?. Action réversible.\\
\verb?Alt+F9? 	&Compose la formule tapée juste avant le curseur avec la fonction \verb?\Cmath{}?. Action réversible.\\
\verb?Ctrl+F9? &	Traduit en LaTeX la formule tapée juste avant le curseur, en mode texte.\\
\verb?Ctrl+Maj+F9? &	Traduit en LaTeX la formule tapée juste avant le curseur, en mode hors-texte.\\
\verb?Ctrl+*? 	&Affiche le symbole ×\\
\verb?Ctrl+/? 	&Affiche le symbole ÷\\
\verb?Ctrl+=? 	&Affiche le symbole environ égal\\
\verb?Ctrl+R? 	&Affiche le symbole √\\
\end{tabular}

Pour configurer TeXworks, voir là : \url{https://code.google.com/p/cmathluatex/}.
\vskip 1em

-- \textbf{Les symboles}

Les symboles suivants sont obtenus en tapant \verb?:? suivi du raccourci correspondant. Je n'indique pas les lettres grecques dans ce tableau ; elles sont toutes disponibles. Par exemple \verb?:de? donne $\Cmath{δ}$. On obtient la version majuscule en tapant le raccourci en majuscule. $\Cmath{Δ}$ est obtenu avec \verb?:DE?.

\begin{tabular}{c c| c c| c c| c c}
Raccourci & Symbole & Raccourci & Symbole & Raccourci & Symbole & Raccourci & Symbole\\
\hline
\verb?:in? & $\Cmath{∞}$ & \verb?:ll? & $\Cmath{ℓ}$ & \verb?:pm? & $\Cmath{±}$ & \verb?:dr? & $\Cmath{∂}$\\
\verb?:vi? & $\Cmath{∅}$ & \verb?:ex? & $\Cmath{∃}$ & \verb?:qs? & $\Cmath{∀}$ & \verb?:e? & $\Cmath{е}$\\
\verb?:i? & $\Cmath{і}$ & \verb?:d? & $\Cmath{:d}$ & \verb?:K? & $\Cmath{𝕂}$ & \verb?:N? & $\Cmath{ℕ}$\\
\verb?:Z? & $\Cmath{ℤ}$ & \verb?:Q? & $\Cmath{ℚ}$ & \verb?:R? & $\Cmath{ℝ}$ & \verb?:C? & $\Cmath{ℂ}$\\
\verb?:Ne? & $\Cmath{ℕe}$ & \verb?:Z? & $\Cmath{ℤe}$ & \verb?:Qe? & $\Cmath{ℚe}$ & \verb?:Re? & $\Cmath{ℝe}$\\
\verb?:Ce? & $\Cmath{ℂe}$ & \verb?:Rm? & $\Cmath{ℝm}$ & \verb?:Rp? & $\Cmath{ℝp}$ & \verb?:Rme? & $\Cmath{ℝme}$\\
\verb?:Rpe? & $\Cmath{ℝpe}$ & \verb?:oij? & $\Cmath{:oij}$ & \verb?:ouv? & $\Cmath{:ouv}$ & \verb?:oijk? & $\Cmath{:oijk}$\\
\end{tabular}
\vskip 1em

Les symboles suivants sont des opérateurs binaires, ils doivent être utilisés entre deux arguments.

\begin{tabular}{c c| c c| c c| c c}
Expression & Affichage & Expression & Affichage & Expression & Affichage & Expression & Affichage\\
\hline
\verb?a:enb? & $\Cmath{a≈b}$ & \verb?a:apb? & $\Cmath{a∈b}$ & \verb?a:asb? & $\Cmath{a⟼b}$ & \verb?a->b? & $\Cmath{a⟶b}$ \\
\verb?a:unb? & $\Cmath{a∪b}$ & \verb?a:itb? & $\Cmath{a∩b}$ & \verb?a:rob? & $\Cmath{a∘b}$ & \verb?a:eqb? & $\Cmath{a:eqb}$\\
\verb?a:cob? & $\Cmath{a≡b}$ & \verb?a:ppb? & $\Cmath{a∨b}$ & \verb?a:pgb? & $\Cmath{a∧b}$ & \verb?a:veb? & $\Cmath{a∧b}$\\
\verb?a:peb? & $\Cmath{a⊥b}$ & \verb?a:sdb? & $\Cmath{a⊕b}$ & \verb?a:npb? & $\Cmath{a∉b}$ & \verb?a:imb? & $\Cmath{a⇒b}$\\
\verb?a:evb? & $\Cmath{a⟺b}$ & \verb?a:rcb? & $\Cmath{a⇐b}$ & \verb?a:icb? & $\Cmath{a⊂b}$ & \verb?a:nib? & $\Cmath{a⊄b}$\\
\end{tabular}

\vskip 1em

-- \textbf{Les fonctions : syntaxes et exemples}

Les arguments notés entre crochets sont facultatifs.

La colonne de gauche contient ce que l'on tape au clavier. La colonne de droite est le résultat de la compilation avec \LuaLaTeX après avoir composition par \verb?\[\Cmath{}\]?. Avec TeXworks, cette opération est réalisée par l'appui sur Maj+F9.

\vskip 1em

\begin{tabular}{m{0.77\linewidth} S{r}}
\multicolumn{2}{>{\columncolor{gray!20}}l}{Reconnaissance des fonctions usuelles} \\
\hline
\verb?f(x)=x*lnx+1? & $\Cmath{ds(f(x)=x*lnx+1)}$\\
\end{tabular}

\begin{tabular}{m{0.7\linewidth} S{r}}
\multicolumn{2}{>{\columncolor{gray!20}}l}{La division : \texttt{n/d}, division en ligne : \texttt{n//d} ou \texttt{n÷d}} \\
\hline
\verb?(x+2)/x? & $\Cmath{ds((x+2)/x)}$\\
\verb?1/(3/4)+1/2=11/2/3? & $\Cmath{ds(1/(3/4)+1/2=11/2/3)}$\\
\verb?1+1/(1+1/(1+...))? & $\Cmath{ds(1+1/(1+1/(1+...)))}$\\
\verb?:e^(:i:pi//4)? & $\Cmath{ds(:e^(:i:pi//4))}$\\
\verb?3÷2? & $\Cmath{ds(3÷2)}$\\

\multicolumn{2}{>{\columncolor{gray!20}}l}{La multiplication implicite, invisible : \texttt{a*b}, visible : \texttt{a×b} ou \texttt{a**b} ou \texttt{a..b}} \\
\hline
\verb?1/2x,√3x,2lnx/x? & $\Cmath{ds(1/2x,√3x,2lnx/x)}$\\
\verb?1/2*x,√3×x,2..lnx/x? & $\Cmath{ds(1/2*x,√3×x,2..lnx/x)}$\\

\multicolumn{2}{>{\columncolor{gray!20}}l}{Gestion des parenthèses inutiles} \\
\hline
\verb?√(x+1),√(x+1)x,(1+n)/3? & $\Cmath{ds(√(x+1),√(x+1)x,(1+n)/3)}$\\

\multicolumn{2}{>{\columncolor{gray!20}}l}{Les racines : \texttt{rac([n,]exp)} ou \texttt{√([n,]exp)}} \\
\hline
\verb?√x,√(3,x)? & $\Cmath{ds(√x,√(3,x))}$\\

\multicolumn{2}{>{\columncolor{gray!20}}l}{Valeur absolue, module : \texttt{abs(exp)}}\\
\hline
\verb?abs(z)? & $\Cmath{abs(z)}$\\

\multicolumn{2}{>{\columncolor{gray!20}}l}{Norme : \texttt{nor(exp)}}\\
\hline
\verb?nor(vec(AB))? & $\Cmath{nor(vec(AB))}$\\

\multicolumn{2}{>{\columncolor{gray!20}}l}{Barre : \texttt{bar(exp)}}\\
\hline
\verb?bar(A:unB)=bar(A):itbar(B)? & $\Cmath{bar(A:unB)=bar(A):itbar(B)}$\\

\multicolumn{2}{>{\columncolor{gray!20}}l}{Tilde : \texttt{til(exp)}}\\
\hline
\verb?til(P:roQ)=til(P):rotil(Q)? & $\Cmath{til(P:roQ)=til(P):rotil(Q)}$\\

\multicolumn{2}{>{\columncolor{gray!20}}l}{Angle : \texttt{ang(exp)}}\\
\hline
\verb?ang((vec(u),vec(v)))? & $\Cmath{ang((vec(u),vec(v)))}$\\

\multicolumn{2}{>{\columncolor{gray!20}}l}{La puissance : \texttt{a\^{}b}}\\
\hline
\verb?:e^(1+1/n),(1/2)^n? & $\Cmath{ds(:e^(1+1/n),(1/2)^n)}$\\
\verb?10^-5,(10^n)^p=10^(n**p)? & $\Cmath{ds(10^-5,(10^n)^p=10^(n**p))}$\\
\verb?x^(2^3)? & $\Cmath{ds(x^(2^3))}$\\

\multicolumn{2}{>{\columncolor{gray!20}}l}{Les indices : \texttt{a\_b}}\\
\hline
\verb?x_1=(1+√5)/2? & $\Cmath{ds(x_1=(1+√5)/2)}$\\
\verb?a=(y_(M_2)-y_(M_1))/(x_(M_2)-x_(M_1))? & $\Cmath{ds(a=(y_(M_2)-y_(M_1))/(x_(M_2)-x_(M_1)))}$\\
\verb?P_1*(X)? & $\Cmath{ds(P_1*(X))}$\\

\multicolumn{2}{>{\columncolor{gray!20}}l}{Les intervalles} \\
\hline
\verb?[0,1/2],]-:in,0]? &  $\Cmath{ds([0,1/2],]-:in,0])}$\\
\verb?[[1,n]]? &  $\Cmath{ds([[1,n]])}$\\
\end{tabular}

\begin{tabular}{m{0.63\linewidth} S{r}}

\multicolumn{2}{>{\columncolor{gray!20}}l}{Forcer le mode textstyle ou displaystyle : \texttt{ts(exp)}, \texttt{ds(exp)}}\\
\hline
\verb?ts(x_1=(1+√5)/2)? & $\Cmath{ts(x_1=(1+√5)/2)}$\\
\verb?ds(x_1=(1+√5)/2)? & $\Cmath{ds(x_1=(1+√5)/2)}$\\

\multicolumn{2}{>{\columncolor{gray!20}}l}{Polices caligraphique : \texttt{cal(exp)}, script : \texttt{scr(exp)}, Zapf Chancery : \texttt{pzc(exp)}}\\
\hline
\verb?cal(M)_n*(:R),scr(C)_f,pzc(E)? & $\Cmath{cal(M)_n*(:R),scr(C)_f,pzc(E)}$\\

\multicolumn{2}{>{\columncolor{gray!20}}l}{Texte : \texttt{"texte"}}\\
\hline
\verb?p="nombre de cas favorables"/"nombre de cas possibles"? & $\Cmath{p="nombre de cas favorables"/"nombre de cas possibles"}$\\

\multicolumn{2}{>{\columncolor{gray!20}}l}{Système : \texttt{sys(expr1[,expr2[,expr3[...])}}\\
\hline
\verb?sys(u_0=1,u_(n+1)=√(u_n+3))? & $\Cmath{sys(u_0=1,u_(n+1)=√(u_n+3))}$\\
\verb?f(x)=sys(2x+1*" si "*x>=0,x^2*" si "*x<0)? & $\Cmath{f(x)=sys(2x+1*" si "*x>=0,x^2*" si "*x<0)}$\\

\multicolumn{2}{>{\columncolor{gray!20}}l}{Accolade : \texttt{acc(exp)}, inférieure : \texttt{aci(exp)}, supérieure : \texttt{acs(exp)}, droite : \texttt{acd(exp)}}\\
\hline
\verb?S=acc(x:ask*x^2,k:ap:R)? & $\Cmath{S=acc(x:ask*x^2,k:ap:R)}$\\
\verb?1+acs(2×3,6)? & $\Cmath{1+acs(2×3,6)}$\\
\verb?1+aci(2×3,6)? & $\Cmath{1+aci(2×3,6)}$\\
\verb?acd(P" est vrai",Q" est vrai"):im(P" et "Q)" est vrai"? & $\Cmath{acd(P" est vrai",Q" est vrai"):im(P" et "Q)" est vrai"}$\\

\multicolumn{2}{>{\columncolor{gray!20}}l}{Vecteur : \texttt{vec(exp)}, vecteur colonne : \texttt{vec(x1,x2[,x3[...])}}\\
\hline
\verb?vec(u)*vec(1,2,3)? & $\Cmath{vec(u)*vec(1,2,3)}$\\
\verb?vec(AB)+vec(BA)=vec(0)? & $\Cmath{vec(AB)+vec(BA)=vec(0)}$\\

\multicolumn{2}{>{\columncolor{gray!20}}l}{Intégrale simple : \texttt{int([inf],[sup],fonction[,variable])}}\\
\hline
\verb?int(0,1,:e^2x,x)=[1/2*:e^2x]_0^1? & $\Cmath{ds(int(0,1,:e^2x,x)=[1/2*:e^2x]_0^1)}$\\
\verb?int([:pi,2:pi],,sinx,x)? & $\Cmath{ds(int([:pi,2:pi],,sinx,x))}$\\
\verb?int(,,lnx)=x*lnx-x+k,k:ap:R? & $\Cmath{ds(int(,,lnx)=x*lnx-x+k,k:ap:R)}$\\

\multicolumn{2}{>{\columncolor{gray!20}}l}{Intégrales double : \texttt{iint([inf],[sup],fonction[,variable1,variable2])}}\\
\hline
\verb?iint(cal(D),,f(x,y),x,y)? & $\Cmath{ds(iint(cal(D),,f(x,y),x,y))}$\\

\multicolumn{2}{>{\columncolor{gray!20}}l}{Intégrales triple : \texttt{iiint([inf],[sup],fonction[,variable1,variable2,variable3])}}\\
\hline
\verb?iiint(cal(V),,f(x,y,z),x,y,z)? & $\Cmath{ds(iiint(cal(V),,f(x,y,z),x,y,z))}$\\

\multicolumn{2}{>{\columncolor{gray!20}}l}{Écrire autour : \texttt{aut(exp,a,b,c,d)}}\\
\hline
\verb?aut(M,1,2,3,4)? & $\Cmath{aut(M,1,2,3,4)}$\\

\multicolumn{2}{>{\columncolor{gray!20}}l}{Somme : \texttt{som([inf],[sup],exp)}}\\
\hline
\verb?som(i=1,+:in,1/2^n)? & $\Cmath{ds(som(i=1,+:in,1/2^n))}$\\

\end{tabular}

\begin{tabular}{m{0.77\linewidth} S{r}}

\multicolumn{2}{>{\columncolor{gray!20}}l}{Produit : \texttt{pro([inf],[sup],exp)}}\\
\hline
\verb?pro(,,(1+1/i))? & $\Cmath{ds(pro(,,(1+1/i)))}$\\

\multicolumn{2}{>{\columncolor{gray!20}}l}{Union : \texttt{uni([inf],[sup],exp)}}\\
\hline
\verb?uni(1<=k<=n,,A_k)? & $\Cmath{ds(uni(1<=k<=n,,A_k))}$\\

\multicolumn{2}{>{\columncolor{gray!20}}l}{Intersection : \texttt{ite([inf],[sup],exp)}}\\
\hline
\verb?ite(i=0,+:in,(]1-a_i,1+b_i[))? & $\Cmath{ds(ite(i=0,+:in,(]1-a_i,1+b_i[)))}$\\

\multicolumn{2}{>{\columncolor{gray!20}}l}{Limite : \texttt{lim(exp1[,exp2],fonction)}}\\
\hline
\verb?lim(x->2,ln(x-2))=-:in? & $\Cmath{ds(lim(x->2,ln(x-2))=-:in)}$\\
\verb?lim(x->√2,x<√2,1/(x^2-2))=+:in? & $\Cmath{ds(lim(x->√2,x<√2,1/(x^2-2))=+:in)}$\\

\multicolumn{2}{>{\columncolor{gray!20}}l}{Sup : \texttt{sup(exp1,exp2)}, idem pour inf, max, min}\\
\hline
\verb?inf(x:ap[a,b],f(x))? & $\Cmath{ds(inf(x:ap[a,b],f(x)))}$\\

\multicolumn{2}{>{\columncolor{gray!20}}l}{Souligner : \texttt{sou(exp)}}\\
\hline
\verb?sou(AB)? & $\Cmath{sou(AB)}$\\

\multicolumn{2}{>{\columncolor{gray!20}}l}{Biffer: \texttt{bif(exp)}}\\
\hline
\verb?(bif(2)×3)/(bif(2)×7)? & $\Cmath{ds((bif(2)×3)/(bif(2)×7))}$\\

\multicolumn{2}{>{\columncolor{gray!20}}l}{Matrice : \texttt{mat(nombre de colonnes,a,b,c,...)}}\\
\hline
\verb?mat(2,1,2,3,4)? & $\Cmath{mat(2,1,2,3,4)}$\\

\multicolumn{2}{>{\columncolor{gray!20}}l}{Déterminant : \texttt{det(nombre de colonnes,a,b,c,...)}}\\
\hline
\verb?det(2,1,2,3,4)? & $\Cmath{det(2,1,2,3,4)}$\\

\multicolumn{2}{>{\columncolor{gray!20}}l}{Crochet : \texttt{cro(nombre de colonnes,a,b,c,...)}}\\
\hline
\verb?cro(2,1,2,3,4)? & $\Cmath{cro(2,1,2,3,4)}$\\

\multicolumn{2}{>{\columncolor{gray!20}}l}{Équivalent à : \texttt{equ(fonction1,point,fonction2)}}\\
\hline
\verb?equ(sinx,0,x)? & $\Cmath{ds(equ(sinx,0,x))}$\\

\multicolumn{2}{>{\columncolor{gray!20}}l}{Tendre vers : \texttt{ten(fonction1,point,fonction2)}}\\
\hline
\verb?ten(lnx,+:in,+:in)? & $\Cmath{ds(ten(lnx,+:in,+:in))}$\\

\multicolumn{2}{>{\columncolor{gray!20}}l}{Dérivée physicienne : \texttt{der(fonction,variable,ordre)}}\\
\hline
\verb?der(f,t,3)? & $\Cmath{ds(der(f,t,3))}$\\

\multicolumn{2}{>{\columncolor{gray!20}}l}{Dérivée partielle : \texttt{derp(fonction,variables)}}\\
\hline
\verb?derp(f,xxyzz)? & $\Cmath{ds(derp(f,xxyzz))}$\\

\multicolumn{2}{>{\columncolor{gray!20}}l}{Petit o : \texttt{pto(point,fonction)}, grand o : \texttt{pto(point,fonction)}}\\
\hline
\verb?x=pto(+:in,x^2)? & $\Cmath{ds(x=pto(+:in,x^2))}$\\
\verb?f=gro(0,x)? & $\Cmath{ds(f=gro(0,x))}$\\

\end{tabular}

\begin{tabular}{m{0.84\linewidth} S{r}}

\multicolumn{2}{>{\columncolor{gray!20}}l}{Restreint à : \texttt{res(fonction,ensemble)}}\\
\hline
\verb?res(f,:Rp)? & $\Cmath{res(f,:Rp)}$\\

\multicolumn{2}{>{\columncolor{gray!20}}l}{Suite : \texttt{sui(nom[,indice])}}\\
\hline
\verb?sui(u)? & $\Cmath{ds(sui(u))}$\\
\verb?sui(u,k)? & $\Cmath{ds(sui(u,k))}$\\
\verb?sui(u,p>=0)? & $\Cmath{ds(sui(u,p>=0))}$\\

\multicolumn{2}{>{\columncolor{gray!20}}l}{Série : \texttt{ser(nom[,indice])}}\\
\hline
\verb?ser(u)? & $\Cmath{ds(ser(u))}$\\
\verb?ser(u,k)? & $\Cmath{ds(ser(u,k))}$\\
\verb?ser(u,p>=0)? & $\Cmath{ds(ser(u,p>=0))}$\\

\end{tabular}

Pour les symboles \LaTeX qui ne sont pas fournis par Cmath\LuaTeX, il suffit de les ajouter dans une expression Cmath\LuaTeX en doublant le backslash. Par exemple : 

\verb?\[\Cmath{mat(3,a_(1,1),...,a_(1,n),\\vdots,\\ddots,\\vdots,a_(n,1),...,a_(n,n))}\]? donnera :

\[\Cmath{mat(3,a_(1,1),...,a_(1,n),\\vdots,\\ddots,\\vdots,a_(n,1),...,a_(n,n))}\]

\vskip 1em

-- \textbf{Calculs formels}

Pour effectuer des calculs formels dans \LuaLaTeX\ ou construire des tableaux de valeurs ou de variations automatiquement, consulter \url{https://github.com/cdevalland/cmathluatex/wiki/Présentation-de-CmathLuaTeX}.


\end{document}
